
\ifdefined\ishandout
\documentclass[11pt,handout,aspectratio=169]{beamer}
\else
\documentclass[11pt,aspectratio=169]{beamer}
\fi

%\documentclass[11pt]{beamer}
\usepackage{mathptmx}
\renewcommand{\sfdefault}{lmss}
\renewcommand{\familydefault}{\sfdefault}
\usepackage[T1]{fontenc}
\usepackage[utf8]{inputenc}
\usepackage{amsmath}
\usepackage{amssymb}
\usepackage{graphicx}
\usepackage{xcolor,multirow,colortbl}
\PassOptionsToPackage{normalem}{ulem}
\usepackage{ulem}
\usepackage{caption}
\captionsetup{labelformat=empty}
\usepackage{bbm}
\usepackage{upgreek}
\usepackage{graphicx}
\setbeamertemplate{section in toc}[sections numbered]
\makeatletter
\usepackage{caption} 
\usepackage{bm}
\usepackage{subfig}
\captionsetup[table]{skip=10pt}
%%%%%%%%%%%%%%%%%%%%%%%%%%%%%% Textclass specific LaTeX commands.
% this default might be overridden by plain title style
\newcommand\makebeamertitle{\frame{\maketitle}}%
% (ERT) argument for the TOC
\AtBeginDocument{%
	\let\origtableofcontents=\tableofcontents
	\def\tableofcontents{\@ifnextchar[{\origtableofcontents}{\gobbletableofcontents}}
	\def\gobbletableofcontents#1{\origtableofcontents}
}

%%%%%%%%%%%%%%%%%%%%%%%%%%%%%% User specified LaTeX commands.
%\documentclass[presentation]{beamer}


\def\Tiny{\fontsize{7pt}{8pt}\selectfont}
\def\Normal{\fontsize{8pt}{10pt}\selectfont}

\usetheme{Madrid}
\usecolortheme{lily}
%\setbeamercovered{transparent}
\useinnertheme{rounded}


\setbeamertemplate{footline}{\hfill\Normal{\insertframenumber/\inserttotalframenumber}}
%\setbeamertemplate{footline}{}

\setbeamertemplate{navigation symbols}{}

\newenvironment{changemargin}[2]{%
	\begin{list}{}{%
			\setlength{\topsep}{0pt}%
			\setlength{\leftmargin}{#1}%
			\setlength{\rightmargin}{#2}%
			\setlength{\listparindent}{\parindent}%
			\setlength{\itemindent}{\parindent}%
			\setlength{\parsep}{\parskip}% 
		}%
		\item[]}{\end{list}}

\setbeamertemplate{footline}{\hfill\insertframenumber/\inserttotalframenumber}
\setbeamertemplate{navigation symbols}{}

%\usepackage{times}  % fonts are up to you
\usepackage{graphicx}
%\usepackage{graphics}
\usepackage{epsfig}
\usepackage{bm}
\usepackage{epsf}
\usepackage{float}
\usepackage[final]{pdfpages}
\usepackage{multirow}
\usepackage{colortbl}
\usepackage{xkeyval}
%\usepackage{sgame}
%\usepackage{pst-node}
\usepackage{listings}
\usepackage{ifthen}
%\usepackage{hyperref}
\usepackage{tikz}

%\usepackage{times}  % fonts are up to you
%\usepackage{graphicx}
%\usepackage{graphics}
\usepackage{epsfig,bm,epsf,float}
\usepackage[final]{pdfpages}
\usepackage{xcolor,multirow,colortbl}
\usepackage{xkeyval}
\usepackage{verbatim}
%\usepackage{sgame}
%\usepackage{pst-node}
\usepackage{listings}
%\usepackage{handoutWithNotes}
%\pgfpagesuselayout{3 on 1 with notes}[letterpaper,border shrink=5mm]
%\pgfpagesuselayout{2 on 1 with notes landscape}[letterpaper,border shrink=5mm]
\usepackage{setspace}
\usepackage{ragged2e}

\setbeamersize{text margin left=1em,text margin right=1em} % CambridgeUS spacing if you use default instead


%\pdfmapfile{+sansmathaccent.map}

% Table formatting
\usepackage{booktabs}


% Decimal align
\usepackage{dcolumn}
\newcolumntype{d}[0]{D{.}{.}{5}}


\global\long\def\expec#1{\mathbb{E}\left[#1\right]}
\global\long\def\var#1{\mathrm{Var}\left[#1\right]}
\global\long\def\cov#1{\mathrm{Cov}\left[#1\right]}
\global\long\def\prob#1{\mathrm{Prob}\left[#1\right]}
\global\long\def\one{\mathbf{1}}
\global\long\def\diag{\operatorname{diag}}
\global\long\def\expe#1#2{\mathbb{E}_{#1}\left[#2\right]}
\DeclareMathOperator*{\plim}{\text{plim}}

%\usefonttheme[onlymath]{serif}

\usepackage{appendixnumberbeamer}
\renewcommand{\thefootnote}{}

\setbeamertemplate{footline}
{
	\leavevmode%
	%   \hbox{%
	%      \begin{beamercolorbox}[wd=\paperwidth,ht=2.25ex,dp=1ex,right]{date in head/foot}%
	%\usebeamerfont{date in head/foot}\insertshortdate{}\hspace*{2em}%
	\hfill
	%turning the next line into a comment, erases the frame numbers
	\insertframenumber{}\hspace*{2ex}\vspace{1ex}
	
	%    \end{beamercolorbox}}%
}

\definecolor{blue}{RGB}{0, 0, 210}
\definecolor{red}{RGB}{170, 0, 0}

\makeatother

\usepackage[english]{babel}

\usepackage{tikz}
\newcommand*\circled[1]{\tikz[baseline=(char.base)]{             \node[circle,ball color=structure.fg, shade,   color=white,inner sep=1.2pt] (char) {\tiny #1};}} 

\makeatletter
\let\save@measuring@true\measuring@true
\def\measuring@true{%
	\save@measuring@true
	\def\beamer@sortzero##1{\beamer@ifnextcharospec{\beamer@sortzeroread{##1}}{}}%
	\def\beamer@sortzeroread##1<##2>{}%
	\def\beamer@finalnospec{}%
}
\makeatother

\definecolor{amethyst}{rgb}{0.6, 0.4, 0.8}

\setbeamersize{text margin left= .8em,text margin right=1em} 
\newenvironment{wideitemize}{\itemize\addtolength{\itemsep}{10pt}}{\enditemize}
\newenvironment{wideitemizeshort}{\itemize}{\enditemize}

\newcommand{\indep}{\perp\!\!\!\!\perp} 


\DeclareMathOperator*{\argmax}{arg\,max}
\DeclareMathOperator*{\argmin}{arg\,min}

\titlegraphic{\vspace{1cm}\footnotesize\textit{*Nota basada en material del curso de Jonathan Roth y Peter Hull en Brown University.}}

\begin{document}
	
	%% Title slide
	\begin{frame}[noframenumbering]{}
		\vspace{0.5cm}
		\title[]{Crash Course: \\ Bases de Estadística y Econometría}
		\author{Paúl J. Corcuera}
		\date{Informática par Economistas \\ Universidad de Piura\\Ciclo 2024-II} 
		\titlepage {\small{}\ }\thispagestyle{empty} \vspace{-30pt}
		
	\end{frame}

\begin{frame}{¿Así que quieres saber de machine learning? Primero las bases }
\begin{center}
\includegraphics[scale=0.4]{stats_bootcamp.png}
\end{center}
\end{frame}


\begin{frame}{¿Qué es Econometría?}

\vspace{0.2cm}
$\rightarrow$ Es la caja de herramientas estadísticas que usamos los economistas para responder \uline{preguntas} economicas usando \uline{data}
\bigskip 

Algunas preguntas que nos interesan: 
\pause
\medskip

\begin{itemize}
\item<1-> ¿Ha aumentado la desigualdad económica desde los años 60?
	\begin{itemize}
		\item<3-> \textbf{Preg. Descriptiva:} pregunta cómo las cosas son (o fueron) en la realidad
	\end{itemize}
\item<1-> ¿Cómo afecta el aumento de salario mínimo en el empleo? 
	\begin{itemize}
		\item<4-> \textbf{Preg. Causal:} ¿Qué hubiera pasado en un mundo contrafactual? 
	\end{itemize}
\item<1->   ¿Cuál va a ser la tasa de desempleo el otro año?
	\begin{itemize}
	\item<5-> \textbf{Preg. Predicción:} ¿Qué pasará el otro año? 
\end{itemize}

\end{itemize}
\medskip

\pause 
\only<6->{
Por lo general, los economistas nos concentramos en las primeras dos, con un énfasis en preguntas causales
}

\end{frame}


\begin{frame}{¿Por qué es difícil responder estas preguntas?}

\begin{wideitemize}
\item
Para preg. descriptivas: solo observamos una \textbf{muestra} de individuos, no la \textbf{población} completa
	\begin{itemize}
		\item 
		Ejemplo: queremos saber la proporción de empleo informal pero solo observamos empleo e informalidad de la Encuesta Nacional de Hogares
	\end{itemize}
\pause

\item El mejor escenario: \\Nuestra muestra es \textbf{aleatoriamente} seleccionada de la población \\
	\begin{itemize}
		\item 
		Por ej., los nombres de los trabajadores que vemos en la encuesta fueron sacados de un sombrero al azar donde estaban todos los posibles trabajadores
		
		\item
		Si este es el caso, debemos tomar en cuenta que por simples chances podemos tener una muestra con diferentes características que la población
	\end{itemize}


\pause
\item El peor escenario: nuestra muestra es \textit{no representativa} de la población que nos interesa
	\begin{itemize}
		\item 
		Por ej., los trabajadores que son formales tenían muchas mayor probabilidad de responder la encuesta 
	\end{itemize}
\end{wideitemize}

\end{frame}


\begin{frame}
\centering
\includegraphics[scale=.3]{dewey-defeats-truman}
\begin{itemize}
	\item 
	En 1948, el Chicago Tribute escribió que Thomas Dewey ganaba a Harry Truman en la elección presidencial, basada en una encuesta a votantes
	
	\pause
	\item
	Pero su encuesta fue por teléfono. En ese año solo la gente rica tenía teléfonos: \\
 muestra $\neq$ población $\rightarrow$ ¡resultados engañosos!
\end{itemize}

\end{frame}

\begin{frame}
\textit{Sesgo de selección} se refiere a contextos como el que mencionamos anteriormente, donde la muestra no es sacada aleatoriamente desde la población de interés

\begin{center}
\includegraphics[width = 0.45\linewidth]{selection_bias_xkcd.png}
\end{center}
\end{frame}


\begin{frame}{¿Por qué responder estás preguntas es difícil? (Parte II)}
\begin{wideitemize}
	\item
	Responder preguntas causales es incluso \textit{más difícil} que las descriptivas. ¿Por qué?
	
	\pause
	\item
	Las preg. causales envuelven tanto un componente descriptivo (¿cómo es el \textit{outcome} en la realidad?) y un componente \textbf{contrafactual} (¿cómo hubieran sido las cosas bajo un tratamiento diferente?)
	
	\pause
	\item
	Ejemplo: ¿Cuál es el efecto causal en sus salarios de ir a UdeP en vez de la U. Pacífico?
		\begin{itemize}
			\item
			Preg. Descriptiva: ¿Cuánto ganan los alumnos de UdeP después de graduarse?
			
			\item
			Preg. Contrafactual: ¿Cuánto \textit{hubieran} ganado los alumnos de UdeP después de graduarse \textit{si hubieran estudiado en la UP}?
		\end{itemize}
		
	\pause 	
	\item Preg. Contrafactuales no se pueden contestar con data solamente. ¡Necesitamos supuestos para aprender de ellos!
	
\end{wideitemize}	
\end{frame}



\begin{frame}{Splitting up the problem}
	\begin{wideitemize}
		
		\item
		When thinking about causal Qs, it's often easier to split the problem in two
		
		\item
		\textbf{Identification:} what could we learn about the parameters we care about (causal effects) if we had the \text{observable data} for the entire population
		\begin{itemize}
			\item 
			Need to make assumptions about how observed outcomes relate to outcomes that would have been realized under different treatments
		\end{itemize}
		
		\item
		\textbf{Statistics}: what can we learn about the full population that we care about from the finite sample that we have?
			\begin{itemize}
				\item 
				Need to understand the process by which our data is generated from the full population
			\end{itemize} 	
		
	\end{wideitemize}	
	
\end{frame}



\begin{frame}{Framework for thinking about these steps}
	
\begin{wideitemize}

\item \textbf{Sample:} the data that you actually observe
	\begin{itemize}
		\item
		A survey of students from Brown and URI graduates about their earnings 
	\end{itemize}

\pause
\item \textbf{Estimator:} a function of the data in the sample 
	\begin{itemize}
		\item Difference in earnings between Brown and URI students in survey
	\end{itemize}

\pause
\item \textbf{Estimand:} a function of the observable data for the \textit{population}
	\begin{itemize}
		\item Difference in earnings between all Brown and URI students 
	\end{itemize}

\pause 
\item \textbf{Target (aka structural) parameter:} what we actually care about
	\begin{itemize}
		\item Causal effect on earnings of going to Brown relative to URI
	\end{itemize}

\end{wideitemize}	

\medskip
\pause
\begin{wideitemize}

\item The process of learning about the \textit{estimand} from the \text{estimator} constructed with your \textit{sample} is called \textbf{statistical estimation/inference}.

\pause 
\item The process of learning about the \textit{parameter} from the \textit{estimand} is called \textbf{identification}.

\end{wideitemize}
	
\end{frame}





\begin{frame}
\vspace{-10mm}
	\includegraphics[width=1.2\textwidth]{BigPicture}
\end{frame}


\begin{frame}{Let's add some math...}
\begin{wideitemize}
	\item Introduce \textbf{potential outcomes} notation
		\begin{itemize}
			\item 
			Super useful framework for thinking about causality! \\
			See the 2021 Nobel Prize writeup on Canvas!
		\end{itemize}
	
	\pause 
	\item $D_i$ = indicator if get treatment (1 if Brown, 0 if URI)
	
	\pause
	\item $Y_i(1)$ = outcome under treament = earnings at Brown
	\item $Y_i(0)$ = outcome under control = earnings at URI
	
	\pause 
	\item Observed outcome $Y_i$ is $Y_i(1)$ if $D_i = 1$ and $Y_i(0)$ if $D_i = 0$. ($Y_i$ is your actual earnings)
	
	\pause
	\item
	We can write the observed outcome as $Y_i = D_i Y_i(1) + (1-D_i) Y_i(0)$
\end{wideitemize}

\end{frame}



\begin{frame}
\begin{wideitemizeshort}
	\item
	Example sample: $(Y_i,D_i)$ for $i=1,...N$. Data with earnings and where you went to school
	
	\pause 
	
	\item
	Example estimator: 
		\begin{itemize}
			\item
			Difference in sample mean of earnings for people who went to Brown and people who went to URI: 
			
			$$ \underbrace{\frac{1}{N_1} \sum_{i:D_i=1} Y_i}_{\text{Avg earnings at Brown in sample}} - \underbrace{\frac{1}{N_0} \sum_{i:D_i=0} Y_i}_{\text{Avg earnings at URI in sample}}$$		
		\end{itemize}
	
	\pause
		\item
		Example estimand: 	
		
				\begin{itemize}
			\item
			Difference in population mean of earnings for people went to Brown and people who went to URI: 
			
			$$\underbrace{ E[ Y_i | D_i = 1] }_{\text{Avg earnings at Brown in population}} - \underbrace{E[Y_i | D_i = 0]}_{\text{Avg earnings at URI in population}}$$
		 
		\end{itemize}
	\pause 
	
	\item
	Example target parameter:
		\begin{itemize}
			\item Causal effect of Brown for Brown students:
			$\underbrace{E[Y_i(1) | D_i =1]}_{\text{Earnings at Brown for Brown students in pop}} - \underbrace{E[Y_i(0) | D_i = 1]}_{\text{Earnings at URI for Brown students in pop}} $.   
		\end{itemize}
\end{wideitemizeshort}
\end{frame}


\begin{frame}{Why is causal identification hard?}
	
	\begin{wideitemize}
		
		\item
		Thought experiment: suppose we had data on earnings for \textit{every} Brown and URI graduate
		
		
		\item
		We can learn from the data:
		$$\color{teal} \underbrace{E[ Y_i(1) | D_i=1  ] }_{\text{Earnings at Brown for Brown Students}} \hspace{0.5cm}  \text{and} \hspace{0.5cm} \underbrace{E[Y_i(0) | D_i = 0]	}_{\text{Earnings at URI for URI students}}$$
		
		\pause 
		\item
		The causal effect of Brown for Brown students is 
		$$ \color{teal}{\underbrace{E[ Y_i(1) | D_i=1  ] }_{\text{Earnings at Brown for Brown Students}}} - \color{red}{ \underbrace{E[ Y_i(0) | D_i=1  ] }_{\text{Earnings at URI for Brown Students}}}$$ 
		
		\pause 
		\item
		The data doesn't tell us $\color{red}{ \underbrace{E[ Y_i(0) | D_i=1 ] }_{\text{Earnings at URI for Brown Students}}}$. Why not? 
		\pause
			\begin{itemize}
				\item 
				Because we never see Brown students going to URI!
			\end{itemize} 
		
	\end{wideitemize}
	
\end{frame}


\begin{frame}
	\begin{wideitemize}
		\item
		One idea to solve this problem would be to assume that: 
		
		$$ \color{red}{ \underbrace{E[ Y_i(0) | D_i=1  ] }_{\text{Earnings at URI for Brown Students}}} = \color{teal}{ \underbrace{E[ Y_i(0) | D_i=0 ] }_{\text{Earnings at URI for URI Students}}}$$
		
		\item
		Why might this give us the wrong answer? 
		
		\pause
		
		\item
		Because Brown students may be different from URI students in other ways that would affect their earnings (regardless of where they went to college)
		
			\begin{itemize}
				\item 
				Academic ability, family background, career goals, etc.
			\end{itemize}
		
		\item
		These differences are referred to as \textit{omitted variables} or \textit{confounding factors}
	\end{wideitemize}
\end{frame}

\begin{frame}{What about experiments?}
\begin{wideitemize}
\item
The gold standard for learning about causal effects is a randomized controlled trial (RCT), aka experiment

\item
Suppose that the Brown and URI administration randomized who got into which college (assume these are the only 2 colleges for simplicity)

\item
Since college is randomly assigned, the only thing that differs between Brown and URI students is the college they went to

\item
Hence, 

		$$ \color{red}{ \underbrace{E[ Y_i(0) | D_i=1  ] }_{\text{Earnings at URI for Brown Students}}} = \color{teal}{ \underbrace{E[ Y_i(0) | D_i=0 ] }_{\text{Earnings at URI for URI Students}}}$$
		
		since we've eliminated any confounding factors


\end{wideitemize}
\end{frame}

\begin{frame}{But running experiments is often hard/impossible}
	\begin{wideitemize}
	
	\item
	Unfortunately, Brown/URI have not let us randomize who gets into which college
		\begin{itemize}
			\item
			At least not yet! If you could convince them to do this, it'd make for a cool senior thesis! 
		\end{itemize}


	\item
	Likewise, it is difficult to convince states to randomize their minimum wages, or other policies
	
	\item
	In some cases, randomization is not just difficult but would be immoral 
	
		\begin{itemize}
			\item 
			``What is the causal effect of spousal death on labor supply?''
		\end{itemize}	
	
	\pause
	\item
	In this course, we'll discuss tools economists try to use when running experiments is not possible.
	\end{wideitemize}
\end{frame}

\begin{frame}{Course Roadmap -- Where we're going}
	\begin{wideitemize}
		\item
		\textbf{Part I ($\sim$ 7 lectures): Review of probability/statistics}. This will give us a mathematical language to talk about:
		
			\begin{enumerate}
				\item 
				\textit{Statistical estimation/inference: }how does the sample we observe relate to the population of interest
				
				\item
				\textit{Identification:} how do observable features of the population relate to (causal) parameters we care about
			\end{enumerate}
		\pause 
		
		\item
		\textbf{Part II ($\sim$ 9 lectures): Linear regression: } We'll discuss ordinarly least squares (OLS), the workhorse model for estimation in econometrics. When does it work, and when will it fail?
		
		\pause
		\item
		\textbf{Part III ($\sim$ 7 lectures:) Other ``quasi-experimental'' strategies}: We'll discuss other strategies for ``mimicking'' an experiment when it's not available, including instrumental variables (IV) and regression discontinuity (RD)
			 
	\end{wideitemize}
\end{frame}

	
	
	\begin{frame}{Outline}

	1. Deriving Multivariate Regression and OLS
	\vspace{0.8cm}
	
	2. Regression and Causality
	\vspace{0.8cm}
	
	3. Regression Odds and Ends
	
	\end{frame}
		
	\begin{frame}{Moving Beyond One ``Regressor''}
		\begin{wideitemize}
			
			\item
			So far we've talked about regression as a way of approximating the CEF $E[Y_i|X_i =x ] \approx \alpha + x \beta $ for a single scalar $X_i$\smallskip
			\begin{itemize}
					\item 
					We then showed how the estimand ($\alpha$,$\beta$) can be estimated by OLS
				\end{itemize}
			
			\pause
			\item
			Next we'll see how this can be generalized to approximate/estimate $E[Y_i| \mathbf{X}_i =\mathbf{x} ] \approx \mathbf{x}' \bm{\beta}$ for a vector $ \mathbf{X_i} = (1,X_{i1},...,X_{iK})'$ 
				\begin{itemize}\smallskip
					\item 
					Note: As usual, I'll be putting vectors/matrices in bold type-face
				\end{itemize}
			
			\pause
			\item
			Two main motivations for this: \medskip
			\end{wideitemize}	

			\pause
\begin{enumerate}
			\item
			We want to use regression to identify causal effects, but conditional unconfoundedness is only plausible with multiple controls \smallskip
			\begin{itemize}
				\item 
				In the Brown/URI example, we may want to control for high school GPA, family income, SAT, race ...
			\end{itemize}
			\medskip
			
			\pause
			\item
			We want a \emph{nonlinear} CEF approx.: e.g. $E[Y_i\mid X_i]\approx\alpha+X_i\beta+X_i^2\gamma$ \smallskip
			\begin{itemize}
			\item We can ``trick`` regression into doing this by setting $\mathbf{X_i} = (1, X_{i}, X_{i}^2)'$
			\end{itemize}
			\end{enumerate}
			
		
		
	\end{frame}
	
		

	
\end{document}


